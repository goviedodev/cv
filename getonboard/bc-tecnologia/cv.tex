\documentclass[11pt,a4paper]{article}
\usepackage[utf8]{inputenc}
\usepackage[T1]{fontenc}
\usepackage[margin=1.5cm]{geometry}
\usepackage{hyperref}
\usepackage{enumitem}
\usepackage{microtype} % Mejora la legibilidad y evita ligaduras problemáticas
\DisableLigatures{encoding = *, family = * } % Crucial para ATS

% Configuración de fuentes
\usepackage{helvet}
\renewcommand{\familydefault}{\sfdefault}

\pagestyle{empty}

\begin{document}

% ENCABEZADO
\begin{center}
    {\LARGE \textbf{Gonzalo Oviedo Lambert}} \\
    \vspace{2pt}
    Limache, Región de Valparaíso, Chile \\
    \vspace{2pt}
    \href{mailto:goviedo.sevenit@gmail.com}{goviedo.sevenit@gmail.com} | +56 9 6372 3603 \\
    \href{https://www.linkedin.com/in/gol}{linkedin.com/in/gol}
\end{center}

% PERFIL PROFESIONAL
\section*{Resumen Profesional}
\hrule
\vspace{5pt}
Ingeniero de Software Full Stack y Arquitecto con experiencia probada en desarrollo web y móvil, operando en entornos multicapa. Dominio de Java/Spring Boot, Node.js y ecosistemas frontend (Javascript, Flutter, React, Vue, HTML, Tailwind CSS, Thymeleaf). Especialista en diseño de microservicios (BFF, API REST), migraciones a la nube (GCP Google) y prácticas DevOps con CI/CD. Destacada capacidad para la resolución ágil de problemas complejos, priorizando soluciones eficientes y promoviendo un sólido trabajo en equipo orientado a resultados.

% EXPERIENCIA ACTUAL (Sugerencia basada en tu actividad de 2026)
\section*{Experiencia Reciente}
\hrule
\vspace{5pt}
\textbf{CTO \& Software Engineer Full Stack} | \textit{Startup de Movilidad} \hfill 2024 -- Presente \\
\begin{itemize}[noitemsep, topsep=0pt]
    \item Desarrollo Full Stack y liderazgo en la construcción de una plataforma de movilidad multicapa (web y móvil).
    \item Implementación de microservicios y APIs REST utilizando Java 21 (Spring Boot) y front end en Flutter.
    \item Prácticas DevOps avanzadas: automatización CI/CD y despliegue de infraestructura en la nube (GCP).
    \item Diseño orientado a objetos, certificaciones de pruebas unitarias/integración y validación en ambientes QA/PROD.
\end{itemize}

\vspace{8pt}

% EXPERIENCIA PASADA (Extracto de tu CV)
\textbf{Java Associative Developer} | \textit{Perficient - Caterpillar} \hfill 2022 -- 2023 \\
\begin{itemize}[noitemsep, topsep=0pt]
    \item Desarrollo y mantenimiento del sistema de e-commerce global usando Java y metodologías ágiles.
    \item Colaboración en equipos multiculturales de EE.UU., India y Latinoamérica.
\end{itemize}

\vspace{8pt}

\textbf{Java Specialist} | \textit{Citibank} \hfill 2021 -- 2022 \\
\begin{itemize}[noitemsep, topsep=0pt]
    \item Desarrollo de soluciones para requerimientos mensuales usando framework interno (Java Bean Spring).
    \item Creación y mantenimiento de reportes para clientes internos.
    \item Migración de base de datos de Sybase a Oracle en equipo multicultural (EE.UU., Ucrania, India, Chile).
    \item Herramientas: Java, Oracle SQL, Spring Beans, Jenkins, Jira, Confluence, Git, Eclipse, Gradle, Websphere.
\end{itemize}

\vspace{8pt}

\textbf{CTO \& Tech Lead} | \textit{Seven IT SpA (Hospital Cruz del Norte, SQM)} \hfill 2017 -- 2020 \\
\begin{itemize}[noitemsep, topsep=0pt]
    \item Arquitectura y desarrollo Full Stack de un Sistema de Gestión en Salud para atención clínica hospitalaria.
    \item Creación de piezas de software para automatizar y mejorar procesos operativos orientados a resultados.
    \item Desarrollo backend y frontend eficiente usando Spring Boot (MVC, REST) y JavaScript (HTML, CSS).
    \item Uso extensivo de PostgreSQL y metodologías de despliegue continuo en Linux y Cloud. 
\end{itemize}

% HABILIDADES (Sin tablas para mejor lectura ATS)
\section*{Habilidades Técnicas}
\hrule
\vspace{5pt}
\textbf{Lenguajes:} Java, JavaScript (ES6+), HTML, CSS, Node.js, Elixir, SQL. \\
\textbf{Frameworks y Patrones:} Spring Boot, Flutter, React, Vue, Tailwind CSS, Thymeleaf, API REST, BFF, MVC. \\
\textbf{Infraestructura y DevOps:} GCP Google, Docker, pipelines CI/CD. \\
\textbf{Bases de Datos y Calidad:} PostgreSQL, Oracle, MongoDB, Pruebas Unitarias e Integración.

% EDUCACIÓN
\section*{Educación}
\hrule
\vspace{5pt}
\textbf{Ingeniería en Ejecución en Computación e Informática} | Universidad del Bío Bío \\
Tesis: Extreme Programming (XP) - Teoría y práctica.

\end{document}
